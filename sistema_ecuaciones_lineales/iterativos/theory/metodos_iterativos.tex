\documentclass[11pt,a4paper]{article}
\usepackage[utf8]{inputenc}
\usepackage[spanish]{babel}
\usepackage{amsmath}
\usepackage{amsfonts}
\usepackage{amssymb}
\usepackage{makeidx}
\usepackage{graphicx}
\usepackage[left=2cm,right=2cm,top=2cm,bottom=2cm]{geometry}
\title{Métodos Iterativos para la Solución de Sistema de Ecuaciones}
%% new enviroments
\newtheorem{mydef}{Definición}[section]
\newtheorem{mytheo}{Teorema}[section]
\newtheorem{mycorol}{Corolario}[mytheo]
% new commands
\newcommand{\re}{\mathbb{R}}
\newcommand{\ds}{\displaystyle}

\begin{document}
\maketitle

\section{Conceptos Previos}
\begin{mydef}[Radio Espectral]
	Sea $A\in\re^{n\times n}$, el radio espectral de $A$ es denotado por $\rho (A)	$ y se define como:
	$$\rho (A) = \max\left\{\vert\lambda\vert /p_{A}(\lambda) = 0\right\}$$\end{mydef}

\begin{mydef}[Diagonalmente dominante]
	Sea $A\in\re^{n\times n}$, $A$ es diagonamlmente dominante si:
		$$\vert a_{ij}\vert > \sum_{j=1, j\neq i}^{n}\vert a_{ij}\vert,\quad i=1-n$$
\end{mydef}

\begin{mytheo}
	La  función llamada radio espectral satisface la ecución:
	$$\rho (A) = \inf_{\Vert .\Vert}\Vert A\Vert$$
	
	En la cual	el infimo se toma sobre todas las normas matriciales subordinadas.
\end{mytheo}

\begin{mytheo}
	Sea las iteraciones $x^{(k)}\in\re^{n\times 1}$ dadas por
	$$x^{(k)} = Gx^{(k-1)}+c$$
	donde $c\in \re^{n\times 1}, G\in\re^{n\times n}$.Dichas iteraciones convergen a $(I-G)^{-1}c$, paar cualquier valor para $x^{(0)}$ si y solo si $\rho (G) < 1$.
\end{mytheo}

\textbf{Prueba}\\
\begin{description}
	\item[$(\Leftrightarrow)$]
		\begin{align*}
			x^{(k)} &= Gx^{(k-1)} +c\\
	 			&= G\left(Gx^{(k-2)}+c\right)+c\\
	 			&\vdots\\
	 			&= G^{k}x^{(0)} + \ds\sum_{i=0}^{k-1}G^{i}c
		\end{align*}
		Como $\rho (G) = \inf_{\Vert .\Vert}\Vert G\Vert < 1$
		$\Rightarrow\exists\Vert .\Vert_{\star}/\Vert G\Vert_{\star} <1$
		$$\Rightarrow\lim_{k\rightarrow\infty}\Vert G\Vert_{\star}^{k} = 0$$
		$$\Rightarrow\lim_{k\rightarrow\infty}\Vert G^{k}x^{(0)}\Vert =0$$
		$\therefore G^{k}x^{(0)}$ es una sucesión convergente a $0$
		De igual modo como $\Vert G\Vert_{\star} < 1$, por la serie de Neuman
		$$(I-G)^{-1} = \sum_{k=0}^{\infty}G^{k}$$
		Por lo tanto la serie $S_{m} = \ds\sum_{i=0}^{m}G^{i}c$ converge a $(I-G)^{-1}c$
		Por lo que $x^{(k)}$ converge, finalmente
		$$\lim_{k\rightarrow\infty}x^{(k)} = (I-G)^{-1}c$$
	\item[$(\Rightarrow)$] 
\end{description}

\begin{mycorol}
	La iteraciones
	\begin{equation}\label{iter}
		Qx^{(k)} = (Q-A)x^{(k)} + b
	\end{equation}
	converge a la solución del sistema $Ax=b$, para cualquier $x^{(0)}$ con la condición $\rho (I-Q^{-1}A) < 1$.
\end{mycorol} 

\textbf{Prueba}\\
Podemos expresar (\ref{iter}) como
$$x^{(k)} = (I-Q^{-1}A)x^{(k-1)} +Q^{-1}b$$
Como $\rho (G) < 1$, entonces $x^{(k)}$ converge a $(I-G)^{-1}c$ con $G=I-Q^{-1}A$ y $c = Q^{-1}b$.Es decir:
\begin{align*}
	x^{(k)} \rightarrow (I-G)^{-1}c &= \left(I-(I-Q^{-1}A)\right)^{-1}\left(Q^{-1}b\right)\\
			&= (Q^{-1}A)^{-1}(Q^{-1}b)\\
			&= (A^{-1}Q)(Q^{-1}b)\\
			&= A^{-1}b = x
\end{align*}
Por los tanto $x^{k}$ converge a la solución del sistema $Ax=b$.
 


\section{Métodos Iterativos}
\subsection{Método de Richardson}
\subsection{Método de Jacobi}
\subsection{Método de Gauss-Seidel}
\subsection{Método de SOR}
\end{document}